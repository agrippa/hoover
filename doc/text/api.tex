\section{API}

\subsection{Vertex APIs}

The core data structure of HOOVER is the graph vertex, represented by objects of
type \texttt{hvr\_sparse\_vec\_t}. Application developers are not expected to
manipulate the internal state of this object directly, but rather using the
following APIs.

Below, several terms may be used to refer to a simulation vertex, including
vertex, sparse vector, or actor.

\subsubsection{\texttt{hvr\_sparse\_vec\_create\_n}}

\begin{verbatim}
hvr_sparse_vec_t *hvr_sparse_vec_create_n(
        const size_t nvecs);
\end{verbatim}

Create \texttt{nvecs} sparse vectors/vertices. This API must be called
collectively on all PEs with the same value of \texttt{nvecs} on each PE. This
will return initialized but empty vertices to the user, to be populated with
initial information.

\subsubsection{\texttt{hvr\_sparse\_vec\_set\_id}}

\begin{verbatim}
void hvr_sparse_vec_set_id(const vertex_id_t id,
        hvr_sparse_vec_t *vec);
\end{verbatim}

Set a globally unique ID \texttt{id} for vertex \texttt{vec} in the current
simulation. It is the programmer's responsibility to provide each vertex a
globally unique ID, and the runtime performs no checks to verify the uniqueness
of each ID. This API must be called once for each vertex in a given simulation
before launching the simulation with the \texttt{hvr\_body} API (described
below).

Vertex IDs do not need to follow any pattern or be contiguous, but application
developers are advised to assign their vertex IDs in such a way that looking up
the PE for a given vertex is a quick, efficient, and preferrably constant time
operation.

\subsubsection{\texttt{hvr\_sparse\_vec\_get\_id}}

\begin{verbatim}
vertex_id_t hvr_sparse_vec_get_id(hvr_sparse_vec_t *vec);
\end{verbatim}

Fetch the globally unique ID assigned to the passed vertex.

\subsubsection{\texttt{hvr\_sparse\_vec\_get\_owning\_pe}}

\begin{verbatim}
int hvr_sparse_vec_get_owning_pe(hvr_sparse_vec_t *vec);
\end{verbatim}

Fetch the PE that owns the passed vertex, i.e. the PE that created and
initialized it. Information on PE ownership is automatically maintained by the
HOOVER runtime. No dynamic load balancing of vertices is performed in HOOVER,
and so vertex ownership is a static and one-to-one relationship.

\subsubsection{\texttt{hvr\_sparse\_vec\_set}}

\begin{verbatim}
void hvr_sparse_vec_set(const unsigned feature,
        const double val, hvr_sparse_vec_t *vec,
        hvr_ctx_t in_ctx);
\end{verbatim}

Set the vertex attribute identified by \texttt{feature} to store the value
\texttt{val} in the vertex \texttt{vec}.

\subsubsection{\texttt{hvr\_sparse\_vec\_get}}

\begin{verbatim}
double hvr_sparse_vec_get(const unsigned feature,
        hvr_sparse_vec_t *vec, hvr_ctx_t in_ctx);
\end{verbatim}

Fetch the value stored for attribute \texttt{feature} in vertex \texttt{vec}. If
the specified attribute was never set on this vertex, the default behavior is to
abort the current simulation.

\subsubsection{\texttt{hvr\_sparse\_vec\_dump}}

\begin{verbatim}
void hvr_sparse_vec_dump(hvr_sparse_vec_t *vec, char *buf,
        const size_t buf_size, hvr_ctx_t in_ctx);
\end{verbatim}

A utility function for creating a human-readable string for the passed vertex
into the provided character \texttt{buf}, which is at least of length
\texttt{buf\_size} bytes. This API can be useful for debugging or other user
diagnostics.

\subsection{Core APIS}

The core of HOOVER is encapsulated in four APIS.

\subsubsection{\texttt{hvr\_ctx\_create}}

\begin{verbatim}
extern void hvr_ctx_create(hvr_ctx_t *out_ctx);
\end{verbatim}

\texttt{hvr\_ctx\_create} initializes the state of a user-allocated HOOVER
context object to be used in later HOOVER user and runtime operations. This API
does not allocate space for the context, the \texttt{out\_ctx} parameter is
expected to point to at least \texttt{sizeof(hvr\_ctx\_t)} bytes of valid
memory. \texttt{out\_ctx} may be on the stack or heap memory segment.

The HOOVER context is used to store several pieces of global state for a given HOOVER
simulation, such as a pointer to the local vertices in the simulation being
managed by the current PE. This should be considered an opaque object by any
application developers, and internal context state should not be directly
manipulated.

HOOVER assumes that the user has already called \texttt{shmem\_init} to
initialize the OpenSHMEM runtime before calling \texttt{hvr\_ctx\_create}.

\subsubsection{\texttt{hvr\_init}}

\begin{verbatim}
extern void hvr_init(const uint16_t n_partitions,
        const vertex_id_t n_local_vertices,
        hvr_sparse_vec_t *vertices,
        hvr_update_metadata_func update_metadata,
        hvr_might_interact_func might_interact,
        hvr_check_abort_func check_abort,
        hvr_vertex_owner_func vertex_owner,
        hvr_actor_to_partition actor_to_partition,
        const double connectivity_threshold,
        const unsigned min_spatial_feature_inclusive,
        const unsigned max_spatial_feature_inclusive,
        const hvr_time_t max_timestep, hvr_ctx_t ctx);
\end{verbatim}

\texttt{hvr\_init} completes initialization of the HOOVER context object by
populating it with several pieces of user-provided information (e.g. callbacks)
and allocating internal data structures. \texttt{hvr\_init} does not launch the
simulation itself, but is the last step before doing so. The arguments passed
are described below:

\begin{enumerate}
    \item \texttt{n\_partitions} - During execution, HOOVER divides the
        simulation space up into partitions as directed by the
        \texttt{actor\_to\_partition} callback (described below). These
        partitions are used to approximately detect interactions between actors
        by first finding actor-to-partition interaction. This is similar but not
        identical to techniques used in the Fast Multipole Method. This argument
        specifies the number of partitions the application developer will use.
    \item \texttt{n\_local\_vertices} - The number of vertices managed by the
        local PE.
    \item \texttt{vertices} - The vertices managed by the local PE. This array
        should be allocated using \texttt{hvr\_sparse\_vec\_create\_n}, and each
        vertex in it should have been initialized using
        \texttt{hvr\_sparse\_vec\_set\_id} and \texttt{hvr\_sparse\_vec\_set}.
    \item \texttt{update\_metadata} - A user callback. On each timestep,
        \texttt{update\_metadata} is passed each local vertex one-by-one along
        with all vertices that the current vertex has edges with (including
        remote vertices). update\_metadata is responsible for updating the local
        state of the current vertex, and deciding if based on those updates any
        remote PEs should become coupled with the current PE's execution.
    \item \texttt{might\_interact} - A user callback. \texttt{might\_interact}
        is called with a single partition ID and a set of partitions to
        determine if a vertex in the provided partition may interact with any
        actor in any partition in the passed set.
    \item \texttt{check\_abort} - A user callback. \texttt{check\_abort}
        is used by the application developer to determine if the current
        simulation should exit based on the state of all local vertices
        following a full timestep. \texttt{check\_abort} also computes any
        local metrics, which are then shared with coupled PEs to compute coupled
        metrics.
    \item \texttt{vertex\_owner} - A user callback. Given a globally unique
        vertex ID, the user is expected to return the PE owning that vertex and
        the offset in that PE's vertices of the specified vertex.
    \item \texttt{actor\_to\_partition} - A user callback. Given a vertex,
        return the partition it belongs in.
    \item \texttt{connectivity\_threshold}, \\
        \texttt{min\_spatial\_feature\_inclusive}, \\
        \texttt{max\_spatial\_feature\_inclusive} - These arguments are all used
        to update edges. Recall that HOOVER automatically updates inter-vertex
        edges based on their ``nearness'' to other vertices in the simulation,
        by some distance measure. Today, that is simply an N-dimensional
        Euclidean distance measure on the features in the range
        [\texttt{min\_spatial\_feature\_inclusive},
        \texttt{max\_spatial\_feature\_inclusive}]. If the computed distance is
        less than \texttt{connectivity\_threshold} those vertices have an edge
        created between them.
    \item \texttt{max\_timestep} - A limit on the number of timesteps for HOOVER
        to run.
    \item \texttt{ctx} - The HOOVER context to initialize. This ctx should
        already have been zeroed using \texttt{hvr\_ctx\_create}.
\end{enumerate}

\subsubsection{\texttt{hvr\_body}}

\begin{verbatim}
extern void hvr_body(hvr_ctx_t ctx);
\end{verbatim}

Launch the simulation problem, as specified by the provided \texttt{ctx}.
\texttt{hvr\_body} only returns when the local PE has completed execution,
either by exceeding the maximum number of timesteps or through a non-zero return
code from the \texttt{check\_abort} callback.

\subsubsection{\texttt{hvr\_finalize}}

\begin{verbatim}
extern void hvr_finalize(hvr_ctx_t ctx);
\end{verbatim}

Perform cleanup of the simulation state. HOOVER assumes that
\texttt{shmem\_finalize} is called after \texttt{hvr\_finalize}.

\subsection{HOOVER Application Skeleton}

Given the above APIs, a standard HOOVER application has the following structure:

\begin{verbatim}
hvr_ctx_t ctx;
hvr_ctx_create(&ctx);

// Create and initialize the vertices in the simulation
hvr_sparse_vec_t *vertices = hvr_sparse_vec_create_n(...);
...

hvr_init(...);

// Launch the simulation
hvr_body(ctx);

// Analyze and display the results of the simulation
...

hvr_finalize(ctx);

\end{verbatim}
