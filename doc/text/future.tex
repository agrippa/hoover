\section{Future Work}

HOOVER is an actively developed project, with several major features on its
roadmap.

First, continued performance analysis and improvement is key as the project
matures. Primarily, this is done through built-in profiling hooks in HOOVER
that offer detailed information on time spent in different application regions.
Additionally, recent work with HOOVER has begun using Jupyter notebooks to
analyze and visualize performance trends in HOOVER executions, which has already
proven valuable. For example, plotting histograms of overall iteration execution
time against histograms of execution time spent in specific code regions
uncovers visual similarities between the overall distribution and the per-region
distribution of execution time. These similarities (e.g. long tails, sharp
Gaussian curves, etc) offer insight into how different code regions positively
or negatively impact overall execution time (both its mean and variance).

Second, explorations of both multi-threaded and multi-GPU execution are on
HOOVER's roadmap. Hybrid distributed and multi-threaded execution will enable
more state to fit in each PE, reducing the amount of communication needed
between PEs.

Finally, we are also interested in exploring a history-less execution mode.
Much of the complexity and overhead of HOOVER results from the need to keep
the history of changes for each actor going back several timesteps, to ensure
that remote PEs can fetch consistent data for their present without seeing into
the ``future''. However, some applications may simply be interested in the
latest state from each PE, regardless of when it was computed. This
simplifies HOOVER's execution and offers new opportunities for optimizations.

In conclusion, HOOVER is a unique, distributed framework for processing of
large-scale and dynamic applications that can be expressed as graph
problems. Unlike other frameworks, it focuses on scale out performance and
datasets that do not fit in CPU or GPU memory. HOOVER's construction on the
OpenSHMEM runtime enables entirely de-coupled, asynchronous execution through
RDMA without the need for interrupting the target PE. This runtime model has
positive long term implications for the scalability of HOOVER, particularly
as hardware platforms grow more parallel and memory access latencies become more
irregular.
